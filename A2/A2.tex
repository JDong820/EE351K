\documentclass{article}
\usepackage{amssymb}
\usepackage{amsmath}
\usepackage{centernot}
\usepackage{graphicx}
\begin{document}
\title{\vspace{-60px}M351K\: Homework 2}
\author{Joshua Dong}
\date{\today}
\maketitle

\section*{1: Disease}
Let $D :=$ the event that a person has the disease.
\\Let $\overline{D} :=$ the event that a person does not have the disease.
\\Let $T :=$ the event that a person tests positive.
\\Let $\overline{T} :=$ the event that a person tests negative.
\\Given $P(D) = 0.0005, P(T|D) = 0.99, P(\overline{T}|\overline{D}) = 0.995$.
\\Then $P(T|\overline{D}) =
1 - P(\overline{T}|\overline{D}) =
1 - 0.995 =
0.005$.
\\$P(T) =
P(T|D)P(D) + P(T|\overline{D})P(\overline{D}) =
0.99(0.0005) + 0.005(1 - 0.0005) =
0.0054925$.
\\Then by Bayes, $P(D|T) =
\frac{P(T|D)P(D)}{P(T)} =
\frac{(0.99)(0.0005)}{0.0054925} =
0.09012289485662267 \approx
9\%$.
\\The probability that a person has the disease given they tested positive is
about 9%. This is a rather low percent, making the test seem inaccurate.
However, the more important percentage is $P(T|D) = 99\%$.

\section*{2: Independence}
\subsection*{a)}
Let $A = B = \mathbb{U}$, where $\mathbb{U}$ is the universe.
\\Then $P(A|B) = P(A) = 1$.
\\Then A and B are pairwise independent.

\subsection*{b)}
A and B are independent. A and C are independent.
\\Then $P(A \cap (B \cup C)) =
P((A \cap B) \cup (A \cap C)) =
P(A \cap B) + P(A \cap C) - P(((A \cap B) \cap (A \cap C))) =
P(A)P(B) + P(A)P(C) - P(A \cap B \cap C) =
P(A)(P(B) + P(C) - \frac{P(A \cap (B \cap C))}{P(A)})$.
\\If $A$ and $B \cap C$ are independent, then 
$P(A \cap (B \cup C)) =
P(A)(P(B) + P(C) - P(B \cap C)) =
P(A)P(B \cup C)$, and A is independent of $B \cup C$.
\\Else, A is not independent of $B \cup C$.
\\Therefore, supposing that A is independent of and A is independent of C
does not imply that A is independent of $B \cup C$ since we never asserted
that $A$ was independent of $B \cap C$.

\subsection*{c)}
If A, B, and C are independent then $P(A \cap B \cap C) = P(A)P(B)P(C)$
and A, B, and C are pairwise independent.
\\Then $P(A \cap (B \cup C))
\\= P((A \cap B) \cup (A \cap C))
\\= P(A \cap B) + P(A \cap C)) - P((A \cap B) \cap (A \cap C))
\\= P(A)P(B) + P(A)P(C) - P(A \cap B \cap C)
\\= P(A)P(B) + P(A)P(C) - P(A)P(B)P(C)
\\= P(A)P(B) + P(A)P(C) - P(A)P(B \cap C)
\\= P(A)(P(B) + P(C) - P(B \cap C))
\\= P(A)P(B \cup C)$.
\\Then A and $B \cup C$ are independent.

\section*{3: Chairs}
Let $S$ be the sample space.
\\Then $S = \{(a,b) \in \mathbb{N} \;|\; a, b \leq n, a \neq b\}$.
\\Let A be the event that $|a - b| \leq 3$.
\\Let $n \geq 7$.
\\Then $P(A)
= P(A \; | \; 3 < a \leq n-3)P(3 < a \leq n-3)
\\+ P(A \; | \; a = 1)P(a = 1)
\\+ P(A \; | \; a = 2)P(a = 2)
\\+ P(A \; | \; a = 3)P(a = 3)
\\+ P(A \; | \; a = n-2)P(a = n-2)
\\+ P(A \; | \; a = n-1)P(a = n-1)
\\+ P(A \; | \; a = n)P(a = n)
\\= P(A \; | \; 3 < a \leq n-3)P(3 < a \leq n-3)
\\+ 2P(A \; | \; a = 1)P(a = 1)
\\+ 2P(A \; | \; a = 2)P(a = 2)
\\+ 2P(A \; | \; a = 3)P(a = 3)
\\= \frac{6}{n-1} \cdot \frac{n-6}{n}
+ \frac{5}{n-1} \cdot \frac{2}{n}
+ \frac{4}{n-1} \cdot \frac{2}{n}
+ \frac{3}{n-1} \cdot \frac{2}{n}
\\= \frac{6}{n-1} \cdot \frac{n-6}{n} +
\frac{2}{n} \cdot (\frac{5}{n-1} 
                 + \frac{4}{n-1} 
                 + \frac{3}{n-1})
\\= \frac{6(n-2)}{(n-1)n}$.
\\If $n = 6$ then $P(A)
= \frac{1}{3} \cdot (\frac{5}{5} 
                   + \frac{4}{5} 
                   + \frac{3}{5})
= \frac{4}{5}$.
\\If $n = 5$ then $P(A)
= \frac{1}{5}
+ \frac{2}{5} \cdot (\frac{4}{4} 
                   + \frac{3}{4})
= \frac{9}{10}$.
\\If $n = 4$ then $P(A) = 1$.
\\If $n = 3$ then $P(A) = 1$.
\\If $n = 2$ then $P(A) = 1$.

\newpage

\section*{4: Set Algebra}
$P(A \cup B \; | \; C)
\\=\frac{P((A \cup B) \cap C)}{P(C)}
\\=\frac{P((A + B - (A \cap B)) \cap C)}{P(C)}
\\=\frac{P(A \cap C) + P(B \cap C) - P((A \cap B) \cap C)}{P(C)}
\\=\frac{P(A \cap C)}{P(C)}
+ \frac{P(B \cap C)}{P(C)}
- \frac{P((A \cap B) \cap C)}{P(C)}
\\=P(A \; | \; C)
+ P(B \; | \; C)
- P(A \cap B \; | \; C)$.

\section*{5: Candy}
We use partitions.
There are 151 choices for each partition, since we may partition
the front or past the end.
Then there are $151^4$ unique possible partitions, since partition positions
may be repeated.
Each unique partition corresponds to a unique color distribution, so there are
a maximum of 519885601 unique jars the factory can produce without repeats.

\section*{6: Coins}
$P($toss 5 = Head$) = P($Head$)
\\= P($Head$ \;|\; $coin a$)P($coin a$) + P($Head$ \;|\; $coin b$)P($coin b$)
\\= 0.7 \cdot \frac{1}{2} + 0.3 \cdot \frac{1}{2}
\\= \frac{1}{2}$.
\\$P($toss 6 = Head$ \;|\; $first 5 tosses are Heads$)
= P($Head$)
= \frac{1}{2}$.

\newpage

\section*{7: Coins}
\subsection*{a)}
Fix one man in one position. This removes repeats due to rotation.
Then there are $7!$ ways to place the rest of the men.
\\For no two women to sit side-by-side, the women must be placed in a seat
between two men.
\\There are 8 positions between the men, and there are $8 \choose 5$ ways to
select seats for the women. Order matters, so there are
$5! \cdot {8 \choose 5}$ ways to seat the women.
\\Then there are a total of $7! \cdot 5! \cdot {8 \choose 5} = 33868800$
unique ways to seat the 13 people independent of rotation.

\subsection*{b)}
There are two couples and 9 people who are not couples.
\\Fix one of the couples to account for rotations.
There are 4 ways to do this.
\\There are then 10 spots for the next couple to be in,
and 2 ways to order the couple.
Thus there are 20 ways to place the second couple.
\\In the remaining 9 slots, there are $9!$ ways to place the remaining people.
\\Then there are a total of $4 \cdot 20 \cdot 9! = 29030400$ ways to seat the 13
people independent of rotation.

\end{document}
