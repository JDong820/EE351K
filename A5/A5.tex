\documentclass{article}
\usepackage{amssymb}
\usepackage{amsmath}
\usepackage{centernot}
\usepackage{graphicx}
\newcommand{\argmax}{\operatornamewithlimits{argmax}}

\begin{document}
\title{\vspace{-60px}M351K\: Homework 5}
\author{Joshua Dong}
\date{\today}
\maketitle

\section*{1: pdf}
$\int_{-\infty}^{\infty}h(x)dx
\\= \int_{-\infty}^{\infty}(\alpha f(x) + (1 - \alpha) g(x))dx
\\= \alpha\int_{-\infty}^{\infty}f(x)dx
+ (1 - \alpha)\int_{-\infty}^{\infty} g(x)dx
\\= \alpha + (1 - \alpha)
\\= 1
$.
\\$\alpha, (1 - \alpha), f(x), g(x) \ge 0 \;\; \forall x \in \mathbb{R}$.
\\Then $\alpha f(x) + (1 - \alpha)g(x) \ge 0 \;\; \forall x \in \mathbb{R}$.
\\Then $h(x) \ge 0 \;\; \forall x \in \mathbb{R}$.
\\Then $h(x)$ is a valid probability density function.

\section*{2: C}
\subsection*{a)}
$1 = \int_{-\infty}^{\infty}f(x)dx = \int_{-\infty}^{\infty}Ce^{-\alpha x}dx$.
\\Then knowing properties of exponential distributions, we conclude that
$C = \alpha$ is the only possibility.
\subsection*{b) plot}
Knowing properties of exponential distributions, we recall
\\$
F(x) = 
\begin{cases}
    1 - e^{-\alpha x} & \text{if }x \ge 0 \\
    0 & \text{if } x < 0
\end{cases}
$.
\subsection*{c)}
$y = F_Y(y) = g(F_Z(z)) = g(1 - e^{-z})$.
\\Then $g(y) =
\begin{cases}
    -\ln{(1-y)} & \text{if } y \in [0, 1] \\
    0 & \text{if } x < 0\\
    \infty & \text{if } x > 1
\end{cases}
$.

\newpage
\section*{3: pdf}
\subsection*{a)}
$F(0) = 0$.
\subsection*{b)}
$F(1)
= \int_0^1{(\frac{3}{2}y - \frac{3}{4}y^2)dx}
= (\frac{3}{4}y^2 - \frac{3}{12}y^3) \Big|_0^1
= \frac{9}{12} - \frac{3}{12}
= \frac{1}{2}$.
\subsection*{c)}
$F(2) = 1$.
\subsection*{d)}
$F(4) = 1$.
\subsection*{maximization}
To find $\argmax{F(a+0.2) - F(a)}$, we can use derivatives.
\\$f(a+0.2) - f(a)
\\= (\frac{3}{2}(a + \frac{1}{5}) - \frac{3}{4}(a + \frac{1}{5})^2)
- (\frac{3}{2}a - \frac{3}{4}a^2)
\\= \frac{3}{2}a + \frac{3}{10}
- \frac{3}{4}a^2 - \frac{6}{20}a - \frac{1}{25}
- \frac{3}{2}a + \frac{3}{4}a^2
\\= \frac{3}{10} - \frac{6}{20}a
\\= \frac{3}{10}(1 - a)$.
\\Convinently, $1 \in [0, 2]$.
\\Since the pdf is non-negative and there is only one critical point,
we know that $\argmax{F(a+0.2) - F(a)} = 1$
\newpage

\section*{4: integration}
\subsection*{integration}
$\displaystyle
1 = \int_1^\infty{\int_1^\infty{ce^{-2(x+y)}dxdy}}
\\= c\int_1^\infty{\int_1^\infty{e^{-2x}e^{-2y}dxdy}}
\\= c\int_1^\infty{e^{-2y}dy(-\frac{1}{2}e^{-2x} \Big|_1^\infty)}
\\= c(0 - (-\frac{1}{2}e^{-2}))\int_1^\infty{e^{-2y}dy}
\\= \frac{1}{2}ce^{-2}(-\frac{1}{2}e^{-2x} \Big|_1^\infty)
\\= \frac{1}{4}ce^{-4}$.
\\Then $c = 4e^4$.
\subsection*{more integration}
$f_X(x) = \int_1^\infty{f_{XY}(x, y)dy}
\\= \int_1^\infty{ce^{-2x}e^{-2y}dy}
\\= ce^{-2x}(-\frac{1}{2}e^{-2y} \Big|_1^\infty)
\\= 2e^{2}e^{-2x}$.
\\By a symmetry argument, we see that $f_Y(y) = 2e^{2}e^{-2y}$.
\\We also notice that $f_{XY}(x, y) = f_Y(y)f_X(x)$.
\\Thus $X$ and $Y$ are independent random variables.

\section*{5: integration}
$\displaystyle
F_X(x)
\\= \int_{-\infty}^x{\frac{\frac{\alpha}{\pi}}{x^2 + \alpha^2}}
\\= \frac{1}{\alpha\pi}\int_{-\infty}^x{\frac{1}{(\frac{x}{\alpha})^2 + 1}}
\\= \frac{1}{\pi}(\tan^{-1}{(\frac{x}{\alpha})} \Big|_{-\infty}^x)
\\= \frac{1}{\pi}\tan^{-1}{(\frac{x}{\alpha})} - (-\frac{1}{2})
\\= \frac{1}{\pi}\tan^{-1}{(\frac{x}{\alpha})} + \frac{1}{2}$.

\section*{6: integration}
\subsection*{a) plot pdf}
$1 = \int_{-\infty}^\infty{f(x)dx} = c\int_0^1{(x - x^3)dx}.
\\= c(\frac{1}{2}x^2 - \frac{1}{4}x^4) \Big|_0^1
\\= \frac{c}{4}$.
\\Then $c = 4$.

\subsection*{b) plot cdf}
$\int_{-\infty}^x{f_X(x)dx} = 
\begin{cases}
    2x^2 - x^4 & \text{if } x \in [0, 1] \\
    0 & \text{if } x < 0\\
    1 & \text{if } x > 1
\end{cases}
$.

\subsection*{c) pdf}
$P(0 < X < 0.5) = F(0.5) - F(0) = F(0.5) = \frac{7}{16}.
\\P(X = 1) = 0.
\\P(0.25 < X < 0.5) = F(0.5) - F(0.25)
= \frac{7}{16} - \frac{31}{256} = \frac{81}{256}
$.

\end{document}
