\documentclass{article}
\usepackage{amssymb}
\usepackage{amsmath}
\usepackage{centernot}
\usepackage{graphicx}
\begin{document}
\title{\vspace{-60px}M351K\: Homework 3}
\author{Joshua Dong}
\date{\today}
\maketitle

\section*{1: Conditional Probability}
Let $P(W)$ be the probability that a white ball is selected and
$P(B_1)$ the probability that the first box is chosen by the coin.
\\$P(B_1 \;|\; W) = \frac{P(B_1)P(W | B_1)}{P(W)}$.
\\$\displaystyle
\frac{P(B_1)P(W | B_1)}{P(W)} =
\frac{(\frac{3}{4})(\frac{7}{13+7})}
     {(\frac{3}{4})(\frac{7}{13+7}) + (\frac{1}{4})(\frac{8}{2+8})} =
\frac{1}{1 + (\frac{8}{40})(\frac{80}{21})} = 
\frac{1}{1 + \frac{16}{21}} =
\frac{21}{37}$.


\section*{2: PMF}
$\displaystyle\sum_{-\infty}^{\infty} p_X(x) = 1$.
\\$\displaystyle\sum_{-\infty}^{\infty} p_X(x) =
\displaystyle\sum_{-\infty}^{1} p_X(x)
+ p_X(0)
+ \displaystyle\sum_{1}^{\infty} p_X(x)
\\= p_X(0) + 2\displaystyle\sum_{x=1}^{\infty} 0.4(1-q)q^{x-1}
\\= b + 0.8(1-q)\displaystyle\sum_{x=1}^{\infty} q^{x-1}
\\= b + 0.8(1-q)(\displaystyle\sum_{x=1}^{\infty} q^{x} + 1)
\textstyle
\\= b + 0.8(1-q)(\frac{p}{1-p} + 1)
\\= b + (\frac{4}{5})(\frac{3}{4})(\frac{4}{3}).
\\1 - (\frac{4}{5}) = b.
\\b = \frac{1}{5}$.
\newpage


\section*{3: Coins}
\subsection*{a)}
The sample space is 
$\{(x_1, x_2, ..., x_{150}) \in \{H, T\}^{150} \;|\;
    x_n \in \{H, T\}\}$.

\subsection*{b)}
Let $h$ be the number of heads.
Then $h \sim \mathcal{B}(150, 0.75)$.


\section*{4: Infinite sums}
Let $d$ be the number of games played. Then $d \sim \mathcal{NB}(1, 0.4)$.
\\The probability that Alice wins the game is
\\$0.2 + 0.2(0.4)^1 + 0.2(0.4)^2 + ...
= 0.2 + 0.2\left(\displaystyle\sum_{n=1}^\infty 0.4^n\right)
\textstyle
= 0.2 + 0.2(\frac{0.4}{1-0.4})
= \frac{1}{3}$.
\\The probability that Bob wins is the probability that Alice
loses:
$1 - \frac{1}{3} = \frac{2}{3}$.

\section*{5: Negative binomial}
$X \sim \mathcal{NB}(n, p)$, where $X$ is the random variable given.

\section*{6: Uniform distribution}
I am quite convinced that Y, Z, and R are discrete random variables.

\subsection*{a)}
$p_Y(x) = 
\begin{cases}
    \frac{3}{8} & \text{if }x = 0 \\
    \frac{1}{16} & \text{if }1 \leq x \leq 10, x \in \mathbb{Z} \\
    0 & \text{if }x < 0 \text{ or }10 < x
\end{cases}
$

\subsection*{b)}
$p_Z(x) =
\begin{cases}
    \frac{5}{8} & \text{if }x = 1 \\
    \frac{1}{16} & \text{if }-5 \leq x \leq 0, x \in \mathbb{Z} \\
    0 & \text{if }x < -5 \text{ or }1 < x
\end{cases}
$

\subsection*{c)}
$p_R(x) =
\begin{cases}
    \frac{1}{8} & \text{if }1 \leq x \leq 5, x \in \mathbb{Z} \\
    \frac{1}{16} & \text{if }x = 0 \text{ or }
        5 < x \leq 10, x \in \mathbb{Z} \\
    0 & \text{if }x < 0 \text{ or }10 < x
\end{cases}
$

\end{document}
